
\documentclass{article}
\usepackage[ascii]{inputenc}
\usepackage[T1]{fontenc}
\usepackage[english]{babel}
\usepackage{amsmath}
\usepackage{amssymb,amsfonts,textcomp}
\usepackage{color}
\usepackage{array}
\usepackage{supertabular}
\usepackage{hhline}
\usepackage[normalem]{ulem}
\usepackage{hyperref}
\usepackage{textcomp}
\usepackage{layout}
\usepackage[numbers]{natbib}
\usepackage{caption}
\usepackage{subcaption}
\usepackage{enumerate}

\hypersetup{colorlinks=true, linkcolor=black, citecolor=black, filecolor=blue, urlcolor=blue, pdftitle=CAD20abstract, pdfauthor=Orest Mykhaskiv, pdfsubject=, pdfkeywords=}
\usepackage[pdftex]{graphicx}

\newcommand\textstyleInternetlink[1]{\textcolor{blue}{#1}}

\setcounter{secnumdepth}{2}
\renewcommand\thesection{\arabic{section}}
\renewcommand\thesubsection{\arabic{section}.\arabic{subsection}}
\makeatletter
\newcommand\arraybslash{\let\\\@arraycr}
\makeatother

\newcommand\liststyleWWviiiNumxxxiii{%
\renewcommand\labelitemi{{\textbullet}}
\renewcommand\labelitemii{{}-}
\renewcommand\labelitemiii{${\blacksquare}$}
\renewcommand\labelitemiv{{\textbullet}}
}
\newcommand\liststyleWWviiiNumxxix{%
\renewcommand\labelitemi{{\textbullet}}
\renewcommand\labelitemii{o}
\renewcommand\labelitemiii{${\blacksquare}$}
\renewcommand\labelitemiv{{\textbullet}}
}
\newcommand\liststyleWWviiiNumxxiv{%
\renewcommand\labelitemi{{\textbullet}}
\renewcommand\labelitemii{o}
\renewcommand\labelitemiii{${\blacksquare}$}
\renewcommand\labelitemiv{{\textbullet}}
}
\newcommand\liststyleWWviiiNumxxi{%
\renewcommand\labelitemi{{\textbullet}}
\renewcommand\labelitemii{o}
\renewcommand\labelitemiii{${\blacksquare}$}
\renewcommand\labelitemiv{{\textbullet}}
}
\newcommand\liststyleWWviiiNumxxiii{%
\renewcommand\labelitemi{{\textbullet}}
\renewcommand\labelitemii{o}
\renewcommand\labelitemiii{${\blacksquare}$}
\renewcommand\labelitemiv{{\textbullet}}
}
\newcommand\liststyleWWviiiNumxv{%
\renewcommand\labelitemi{{\textbullet}}
\renewcommand\labelitemii{o}
\renewcommand\labelitemiii{${\blacksquare}$}
\renewcommand\labelitemiv{{\textbullet}}
}
\newcommand\liststyleWWviiiNumxxx{%
\renewcommand\theenumi{\arabic{enumi}}
\renewcommand\theenumii{\alph{enumii}}
\renewcommand\theenumiii{\roman{enumiii}}
\renewcommand\theenumiv{\arabic{enumiv}}
\renewcommand\labelenumi{[\theenumi]}
\renewcommand\labelenumii{\theenumii.}
\renewcommand\labelenumiii{\theenumiii.}
\renewcommand\labelenumiv{\theenumiv.}
}
\newcommand\liststyleWWviiiNumxxxii{%
\renewcommand\labelitemi{{\textbullet}}
\renewcommand\labelitemii{o}
\renewcommand\labelitemiii{${\blacksquare}$}
\renewcommand\labelitemiv{{\textbullet}}
}

\renewcommand\refname{}
\renewcommand{\bibsection}{}
\renewcommand{\theequation}{2.\arabic{equation}}


\setlength\paperwidth{19.05cm}
\setlength\paperheight{26.162cm}
\setlength\voffset{-1in}
\setlength\hoffset{-1in}
\setlength\topmargin{30pt}
\setlength\oddsidemargin{1.524cm}
\setlength\textheight{576pt}
\setlength\textwidth{16.001999cm}
\setlength\footskip{0.841cm}
\setlength\headheight{16pt}
\setlength\headsep{28pt}


\usepackage{fancyhdr}
\pagestyle{fancy}
\fancyhf{}
\renewcommand{\headrulewidth}{0pt}
\renewcommand{\footrulewidth}{0.4pt}
\fancyhead[R]{\thepage}
\fancyfoot[R]{Proceedings of CAD'20, Barcelona, Spain, July 6-8, 2020, aaa-bbb\\ {\footnotesize{\textcopyright}} 2020 CAD Solutions, LLC, \ULurl{http://www.cad-conference.net}}

\usepackage{etoolbox}
\patchcmd{\thebibliography}{\section*{\refname}}{}{}{}
\captionsetup[figure]{labelformat={default},name={Fig.}}

\setlength{\bibsep}{2pt plus 0.3ex}
\setcounter{page}{1}

\makeatletter
\DeclareUrlCommand\ULurl@@{%
  \def\UrlFont{\ttfamily\color{blue}}%
  \def\UrlLeft{\uline\bgroup}%
  \def\UrlRight{\egroup}}
\def\ULurl@#1{\hyper@linkurl{\ULurl@@{#1}}{#1}}
\DeclareRobustCommand*\ULurl{\hyper@normalise\ULurl@}
\makeatother

\begin{document}

{\centering  \includegraphics[width=5.173cm,height=2.193cm]{images/CADconverted-img001.jpg} \par}

\vspace{5pt}
\noindent
\underline{Title:}

\noindent{\bfseries
CAD-compatible conversion of multi-sided surfaces -- issues and insights }

\vspace{1em}
\noindent \underline{Authors:}
\newline
P\'eter Salvi, salvi@iit.bme.hu, Budapest University of Technology and Economics \newline
Tam\'as V\'arady, varady@iit.bme.hu, Budapest University of Technology and Economics \newline
Alyn Rockwood, alynrock@gmail.com, Boulder Graphics LLC, USA

\vspace{1em}
\noindent \underline{Keywords:}\newline
B\'ezier surface, Trimming, Charrot--Gregory patch, S-patch


\bigskip


\noindent \underline{DOI:} 10.14733/cadconfP.2020.xxx-yyy

\vspace{10pt}
\noindent\underline{Introduction:}\vspace{0.2em}\newline
The majority of aesthetic objects around us is represented by free-form shapes, that
naturally involves modeling multi-sided (i.e., non-four-sided) surfaces, as well.  The
mathematical representation of such patches is still an active area in CAGD, and although
a great variety of approaches have been published, none of the genuine $n$-sided
formulations has been standardized so far.

On the other hand, commercial CAD/CAM systems and related application programs only accept
data in standard formats, such as tensor product NURBS surfaces. For this reason, it is a
widely applied practice to convert multi-sided surfaces into a CAD-compatible
representation either by (i)~approximating them with larger quadrilaterals,
\emph{trimming} away the exterior part beyond the boundaries, or (ii)~\emph{splitting}
them into smaller four-sided patches.

Both techniques have their deficiencies. They only approximate the original multi-sided
patch, and with trimming even $C^0$ continuity between the adjacent patches cannot be
ensured (in general). In the splitting scheme the subdividing curves in the interior
weaken the overall continuity of the surface.

Ideally, we would like to have an $n$-sided patch that
\begin{enumerate}[i)]
\item can be used for design (has intuitive controls),
\item can be attached to adjacent patches with $G^1$ or higher continuity,
\item and can also be represented \emph{accurately} as a tensor product NURBS surface.
\end{enumerate}

The above problem can be resolved, if the multi-sided surface can be represented as a
rational polynomial of two parametric variables. Then it can directly be converted into
NURBS form, without either changing the surface or harming continuity. The result will be
a trimmed surface with exact trimming curves.
 
Some of the well-known multi-sided schemes allow computing a trimmed bi-parametric
representation. Our goal in this paper is to review these and discuss the difficulties of
the conversion process. We are going to provide further insights into specific
computational and geometric problems, that have not been discussed elsewhere and are
useful for analyzing the ``pros and cons'' of these representations. Four schemes will be
described, and a general discussion concludes the paper.

\vspace{10pt}
\noindent\underline{S-patch:}\vspace{0.2em}\newline
The S-patch of Loop \& DeRose~\cite{spatch1} is a generalization of the B\'ezier triangle, or,
more precisely, a B\'ezier simplex mapping from $(n-1)$D to 3D, where the $n$ coordinates of the
domain are supplied by generalized barycentric coordinates. The \emph{depth} ($d$) of an S-patch
is the number of deCasteljau steps it takes to evaluate a surface point, i.e., something similar
to the degree of a B\'ezier triangle (but not the degree of the S-patch itself).
S-patches have many nice properties, and are known to be convertable into tensor product
rational B\'ezier surfaces of degree $d(n-2)$.

One drawback of this representation is its large number of control points, which renders it
inconvenient for interactive design. For example, a five-sided patch of depth 5 has 126 control
points, while a 4-sided tensor product patch has only 36. A possible workaround is to use a
$G^1$ \emph{frame} for design that defines the tangent planes at the boundaries.
After increasing the depth by 3, these boundary constraints can be interpolated~\cite{spatch2},
and the remaining interior control points can be set by some heuristic to generate a smooth
surface~\cite{salvi-kepaf}, see the figure below.
\begin{figure}[h!]
  \centering
  \includegraphics[width = 0.3\textwidth]{images/5-5-bezier-ribbon.png}
  \hfill
  \includegraphics[width = 0.3\textwidth]{images/5-5-cnet-ribbon.png}
  \hfill
  \includegraphics[width = 0.3\textwidth]{images/5-5-cnet-full.png}
  %% \caption{Creating an S-patch based on a $G^1$ frame.}
  %% \label{fig:spatch}
\end{figure}

The CAD-compatible conversion presented in~\cite{spatch1} is a two-step process:
first convert the surface to a
four-sided S-patch, and then to a tensor product patch. The first step is based on the
composition of B\'ezier simplexes%~\cite{simplex1}
, which has very high complexity.
Even using a more efficient algorithm~\cite{simplex2}, converting a modest-sized S-patch
still requires minutes of computation on today's machines~\cite{salvi-wait}.

Here we propose an alternative conversion process. Since a B\'ezier simplex is just a polynomial,
the only problem is how to express the generalized barycentric coordinates as a rational polynomial
of the $(u,v)$ parameters on the 2D domain. Using Wachspress coordinates over a regular $n$-sided
polygon, the barycentric coordinates $\{\lambda_i\}$ are expressed as
\begin{equation}
  \label{eq:wachspress}
  \lambda_i(u,v) = \prod_{\substack{j=1\\j\notin\{i-1,i\}}}^nh_j(u,v) \quad\bigg/\quad
                   \sum_{k=1}^n\prod_{\substack{j=1\\j\notin\{k-1,k\}}}^nh_j(u,v),
\end{equation}
where the indexing is cyclic, and $h_j(u,v)$ is some distance
from the $j$-th side of the domain polygon. The implicit
equation of the line containing this side is such a distance function, and is a linear polynomial,
so the Wachspress coordinates can be expressed as rational polynomials of degree $n-2$. We
normalize the distances such that they take on take on the value 1 at vertices adjacent to the side.

Using this method, the B\'ezier control points of the tensor product representation
can be located by straightforward computation, which takes only milliseconds.

\vspace{10pt}
\noindent\underline{Warren's patch:}\vspace{0.2em}\newline
Warren~\cite{warren} created multi-sided patches from B\'ezier triangles by assigning $0/0$
\emph{base points} to some of the control points, essentially cutting off the corners, and thus
creating 5- and 6-sided surfaces. A simple conversion to a (degenerate) tensor product form is
also shown in the paper.

A nice property of this patch is that the ``remaining'' vertices define the behavior of the
boundary in the same way as in a normal B\'ezier triangle, i.e., the first control row defines
its position as a B\'ezier curve, the second its first derivatives etc.
The figure below shows the creation of a 5-sided patch and its conversion.
\begin{figure}[h!]
  \centering
  \includegraphics[width = 0.20\textwidth]{images/warren-cnet.png}
  \hspace{3cm}
  \includegraphics[width = 0.20\textwidth]{images/warren-quad.png}
  %% \begin{subfigure}{0.45\textwidth}
  %%   \centering
  %%   \includegraphics[width = 0.6\textwidth]{images/warren-cnet.png}
    %% \caption{Creating a 5-sided patch}
    %% \label{fig:warren-cnet}
  %% \end{subfigure}
  %% \begin{subfigure}{0.45\textwidth}
  %%   \centering
  %%   \includegraphics[width = 0.6\textwidth]{images/warren-quad.png}
    %% \caption{Degenerate tensor product patch}
    %% \label{fig:warren-quad}
  %% \end{subfigure}
  %% \caption{Warren's patch.}
  %% \label{fig:warren}
\end{figure}

Note, however, that not all degree configurations are available. A 6-sided patch with degree-$d$
boundaries can be created from a triangle of degree $3d$, but due to its asymmetric construction,
a 5-sided patch cannot have boundaries of the same degree. Moreover, using control points with zero
weight is not a standard practice, and is not supported by many systems. Meshing
also presents a problem, as a uniform grid on the domain would result in distorted triangles.

% Toric patches ...

\vspace{10pt}
\noindent\underline{Kato's patch:}\vspace{0.2em}\newline
Kato~\cite{kato} proposed a surface as the transfinite interpolation of boundary curves with
cross-derivatives. When these boundary constraints are given as a $G^1$ frame (and hence
are polynomial), the whole patch may become polynomial. The tricky part is the parameterization:
this representation uses two parameters, a \emph{side parameter} $s_i$
that takes on values from 0 to 1 as it sweeps from one adjacent side to the other,
and a \emph{distance parameter} $h_i$ that vanishes on the base side, see
Figure~\ref{fig:parameters}.
\begin{figure}[b!]
  \begin{subfigure}{0.30\textwidth}
    \includegraphics[width = \textwidth]{images/domain.pdf}
    \caption{Domain}
    \label{fig:domain}
  \end{subfigure}
  \hfill
  \begin{subfigure}{0.30\textwidth}
    \begin{minipage}[b][5cm][b]{\textwidth}
      \centering
      \includegraphics[width = 0.83\textwidth]{images/s-params.pdf}
      \vspace*{9mm}
    \end{minipage}
    \caption{$s_i$ isolines for two sides}
    \label{fig:s}
  \end{subfigure}
  \hfill
  \begin{subfigure}{0.30\textwidth}
    \begin{minipage}[b][5cm][b]{\textwidth}
      \centering
      \includegraphics[width = 0.83\textwidth]{images/h-params.pdf}
      \vspace*{9mm}
    \end{minipage}
    \caption{$h_i$ isolines for two sides}
    \label{fig:h}
  \end{subfigure}
  \caption{Parameterization.}
  \label{fig:parameters}
\end{figure}

The distance parameters can be computed the same way as in the creation of S-patches; for side
parameters we can use $s_i=h_{i-1}/(h_{i-1}+h_{i+1})$, which gives a rational polynomial
representation.

Kato's patch is easily extendable to handle $G^2$ continuity, if the side constraints have 3
control rows (so the cross-degree $d^\perp$ is 2 instead of 1).
Then the whole surface becomes a rational tensor product B\'ezier patch of
degree $nd+(n-1)(d^\perp+1)+d^\perp$.

\vspace{10pt}
\noindent\underline{Charrot--Gregory patch:}\vspace{0.2em}\newline
The same idea can be used to convert Charrot--Gregory patches~\cite{charrot} that use
only side parameters. (Note that on a regular domain the $s_i$ parameters defined above
will be the same as the radial parameterization in the original paper.)
The converted patch will be of degree $nd+2(n-2)$.

For triangular surfaces, we can use the $h_{i-1}$ parameter as a side parameter for the $i$-th
side instead, which is not rational, so the overall degree is reduced to $d+3$.
An example with mean curvature and isophote lines is shown below.
\begin{figure}[h!]
  \includegraphics[width = 0.2139\textwidth]{images/trebol3-cnet.png}
  \hfill
  \includegraphics[width = 0.23\textwidth]{images/trebol3-contour.jpg}
  \hfill
  \includegraphics[width = 0.23\textwidth]{images/trebol3-mean-iso.jpg}
  \hfill
  \includegraphics[width = 0.23\textwidth]{images/trebol3-zebra.jpg}
  %% \begin{subfigure}{0.23\textwidth}
  %%   \centering
  %%   \includegraphics[width = 0.93\textwidth]{images/trebol3-cnet.png}
  %%   \caption{Frame and control net}
  %%   \label{fig:trebol-cnet}
  %% \end{subfigure}
  %% \begin{subfigure}{0.23\textwidth}
  %%   \includegraphics[width = \textwidth]{images/trebol3-contour.jpg}
  %%   \caption{Contouring}
  %%   \label{fig:trebol-contour}
  %% \end{subfigure}
  %% \begin{subfigure}{0.23\textwidth}
  %%   \includegraphics[width = \textwidth]{images/trebol3-mean-iso.jpg}
  %%   \caption{Mean curvature}
  %%   \label{fig:trebol-mean}
  %% \end{subfigure}
  %% \begin{subfigure}{0.23\textwidth}
  %%   \includegraphics[width = \textwidth]{images/trebol3-zebra.jpg}
  %%   \caption{Isophote lines}
  %%   \label{fig:trebol-iso}
  %% \end{subfigure}
  %% \caption{Conversion of Charrot--Gregory patches.}
  %% \label{fig:trebol}
\end{figure}

\vspace{10pt}
\noindent\underline{Discussion:}\vspace{0.2em}\newline
Blabla
\begin{figure}[h!]
  \centering
  \includegraphics[width = 0.9\textwidth]{images/rotations.png}
  \caption{Effects of rotating the multi-sided domain.}
  \label{fig:rotations}
\end{figure}
\begin{figure}[h!]
  \centering
  \begin{subfigure}{0.3\textwidth}
    \includegraphics[width = \textwidth]{images/8sided-cnet.png}
    \caption{Frame and control net}
    \label{fig:8sided-cnet}
  \end{subfigure}
  \hspace{3cm}
  \begin{subfigure}{0.3\textwidth}
    \includegraphics[width = \textwidth]{images/8sided-iso.png}
    \caption{Isophote lines}
    \label{fig:8sided-iso}
  \end{subfigure}

  \caption{Using a larger multi-sided domain.}
  \label{fig:8sided}
\end{figure}

\vspace{1em}
\noindent\underline{Conclusions:}\vspace{0.2em}\newline
Blabla

\vspace{1em}
\noindent\underline{Acknowledgement:}\vspace{0.2em}\newline
This project has been supported by the Hungarian Scientific Research Fund (OTKA, No.~124727)
and the National R\&D and Innovation Fund
(TUDFO/51757/2019-ITM, Thematic Excellence Program).

\vspace{1em}
\noindent\underline{References:}\vspace{-1.9em}\newline
\renewcommand{\section}[2]{}
\begin{thebibliography}{99}

\bibitem{charrot} Charrot, P.; Gregory, J.~A.: A pentagonal surface patch for computer aided geometric design, Computer Aided Geometric Design, 1(1), 1984, 87--94.\\\ULurl{https://doi.org/10.1016/0167-8396(84)90006-2}
%\bibitem{simplex1} DeRose, T.~D.: Composing B\'ezier simplexes, ACM Transactions on Graphics, 7(3), 1988, 198--221.\\\ULurl{https://doi.org/10.1145/44479.44482}
\bibitem{simplex2} DeRose, T.~D.; Goldman, R.~N.; Hagen, H.; Mann, S.: Functional composition algorithms via blossoming, ACM Transactions on Graphics, 12(2), 1993, 113--135.\\\ULurl{https://doi.org/10.1145/151280.151290}
%\bibitem{floater} Floater, M.~S.: Generalized barycentric coordinates and applications, Acta Numerica, 24, 2015, 161--214.
\bibitem{kato} Kato, K.: Generation of $n$-sided surface patches with holes, Computer-Aided Design, 23(10), 1991, 676--683. \ULurl{https://doi.org/10.1016/0010-4485(91)90020-W}
%\bibitem{toric} Krasauskas, R.: Toric surface patches, Advances in Computational Mathematics, 17(1), 2002, 89--113. \ULurl{https://doi.org/10.1023/A:1015289823859}
\bibitem{spatch1} Loop, Ch.~T.; DeRose, T.~D.: A multisided generalization of B\'ezier surfaces, ACM Transactions on Graphics, 8(3), 1989, 204--234. \ULurl{https://doi.org/10.1145/77055.77059}
\bibitem{spatch2} Loop, Ch.~T.; DeRose, T.~D.: Generalized B-spline surfaces of arbitrary topology, ACM, 17th Conference on Computer Graphics and Interactive Techniques, 1990, 347--356.\\\ULurl{https://doi.org/10.1145/97879.97917}
\bibitem{salvi-kepaf} Salvi, P.: $G^1$ hole filling with S-patches made easy, Proceedings of the 12th Conference of the Hungarian Association for Image Processing and Pattern Recognition, 2019, 1--8.
\bibitem{salvi-wait} Salvi, P.: On the CAD-compatible conversion of S-patches, Proceedings of the Workshop on the Advances of Information Technology, 2019, 72--76.
%\bibitem{transfinite} V\'arady, T.; Rockwood, A.; Salvi, P.: Transfinite surface interpolation over irregular $n$-sided domains, Computer-Aided Design, 43(11), 2011, 1330--1340. \ULurl{https://doi.org/10.1016/j.cad.2011.08.028}
\bibitem{warren} Warren, J.: Creating multisided rational B\'ezier surfaces using base points, ACM Transactions on Graphics, 11(2), 1992, 127--139. \ULurl{https://doi.org/10.1145/130826.130828}

\end{thebibliography}

\end{document}
