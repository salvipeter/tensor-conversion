
\documentclass{article}
\usepackage[ascii]{inputenc}
\usepackage[T1]{fontenc}
\usepackage[english]{babel}
\usepackage{amsmath}
\usepackage{amssymb,amsfonts,textcomp}
\usepackage{color}
\usepackage{array}
\usepackage{supertabular}
\usepackage{hhline}
\usepackage[normalem]{ulem}
\usepackage{hyperref}
\usepackage{textcomp}
\usepackage{layout}
\usepackage[numbers]{natbib}
\usepackage{caption}
\usepackage{subcaption}
\usepackage{enumerate}

\hypersetup{colorlinks=true, linkcolor=black, citecolor=black, filecolor=blue, urlcolor=blue, pdftitle=CAD20abstract, pdfauthor=Orest Mykhaskiv, pdfsubject=, pdfkeywords=}
\usepackage[pdftex]{graphicx}

\newcommand\textstyleInternetlink[1]{\textcolor{blue}{#1}}

\setcounter{secnumdepth}{2}
\renewcommand\thesection{\arabic{section}}
\renewcommand\thesubsection{\arabic{section}.\arabic{subsection}}
\makeatletter
\newcommand\arraybslash{\let\\\@arraycr}
\makeatother

\newcommand\liststyleWWviiiNumxxxiii{%
\renewcommand\labelitemi{{\textbullet}}
\renewcommand\labelitemii{{}-}
\renewcommand\labelitemiii{${\blacksquare}$}
\renewcommand\labelitemiv{{\textbullet}}
}
\newcommand\liststyleWWviiiNumxxix{%
\renewcommand\labelitemi{{\textbullet}}
\renewcommand\labelitemii{o}
\renewcommand\labelitemiii{${\blacksquare}$}
\renewcommand\labelitemiv{{\textbullet}}
}
\newcommand\liststyleWWviiiNumxxiv{%
\renewcommand\labelitemi{{\textbullet}}
\renewcommand\labelitemii{o}
\renewcommand\labelitemiii{${\blacksquare}$}
\renewcommand\labelitemiv{{\textbullet}}
}
\newcommand\liststyleWWviiiNumxxi{%
\renewcommand\labelitemi{{\textbullet}}
\renewcommand\labelitemii{o}
\renewcommand\labelitemiii{${\blacksquare}$}
\renewcommand\labelitemiv{{\textbullet}}
}
\newcommand\liststyleWWviiiNumxxiii{%
\renewcommand\labelitemi{{\textbullet}}
\renewcommand\labelitemii{o}
\renewcommand\labelitemiii{${\blacksquare}$}
\renewcommand\labelitemiv{{\textbullet}}
}
\newcommand\liststyleWWviiiNumxv{%
\renewcommand\labelitemi{{\textbullet}}
\renewcommand\labelitemii{o}
\renewcommand\labelitemiii{${\blacksquare}$}
\renewcommand\labelitemiv{{\textbullet}}
}
\newcommand\liststyleWWviiiNumxxx{%
\renewcommand\theenumi{\arabic{enumi}}
\renewcommand\theenumii{\alph{enumii}}
\renewcommand\theenumiii{\roman{enumiii}}
\renewcommand\theenumiv{\arabic{enumiv}}
\renewcommand\labelenumi{[\theenumi]}
\renewcommand\labelenumii{\theenumii.}
\renewcommand\labelenumiii{\theenumiii.}
\renewcommand\labelenumiv{\theenumiv.}
}
\newcommand\liststyleWWviiiNumxxxii{%
\renewcommand\labelitemi{{\textbullet}}
\renewcommand\labelitemii{o}
\renewcommand\labelitemiii{${\blacksquare}$}
\renewcommand\labelitemiv{{\textbullet}}
}

\renewcommand\refname{}
\renewcommand{\bibsection}{}
\renewcommand{\theequation}{2.\arabic{equation}}


\setlength\paperwidth{19.05cm}
\setlength\paperheight{26.162cm}
\setlength\voffset{-1in}
\setlength\hoffset{-1in}
\setlength\topmargin{30pt}
\setlength\oddsidemargin{1.524cm}
\setlength\textheight{576pt}
\setlength\textwidth{16.001999cm}
\setlength\footskip{0.841cm}
\setlength\headheight{16pt}
\setlength\headsep{28pt}


\usepackage{fancyhdr}
\pagestyle{fancy}
\fancyhf{}
\renewcommand{\headrulewidth}{0pt}
\renewcommand{\footrulewidth}{0.4pt}
\fancyhead[R]{\thepage}
\fancyfoot[R]{Proceedings of CAD'20, Barcelona, Spain, July 6-8, 2020, aaa-bbb\\ {\footnotesize{\textcopyright}} 2020 CAD Solutions, LLC, \ULurl{http://www.cad-conference.net}}

\usepackage{etoolbox}
\patchcmd{\thebibliography}{\section*{\refname}}{}{}{}
\captionsetup[figure]{labelformat={default},name={Fig.}}

\setlength{\bibsep}{2pt plus 0.3ex}
\setcounter{page}{1}

\makeatletter
\DeclareUrlCommand\ULurl@@{%
  \def\UrlFont{\ttfamily\color{blue}}%
  \def\UrlLeft{\uline\bgroup}%
  \def\UrlRight{\egroup}}
\def\ULurl@#1{\hyper@linkurl{\ULurl@@{#1}}{#1}}
\DeclareRobustCommand*\ULurl{\hyper@normalise\ULurl@}
\makeatother

\begin{document}

{\centering  \includegraphics[width=5.173cm,height=2.193cm]{images/CADconverted-img001.jpg} \par}

\vspace{5pt}
\noindent
\underline{Title:}

\noindent{\bfseries
CAD-compatible conversion of multi-sided surfaces -- issues and insights }

\vspace{1em}
\noindent \underline{Authors:}
\newline
P\'eter Salvi, salvi@iit.bme.hu, Budapest University of Technology and Economics \newline
Tam\'as V\'arady, varady@iit.bme.hu, Budapest University of Technology and Economics \newline
Alyn Rockwood, alynrock@gmail.com, Boulder Graphics LLC, USA

\vspace{1em}
\noindent \underline{Keywords:}\newline
B\'ezier surface, Trimming, Charrot--Gregory patch, S-patch


\bigskip


\noindent \underline{DOI:} 10.14733/cadconfP.2020.xxx-yyy

\vspace{10pt}
\noindent\underline{Introduction:}\vspace{0.2em}\newline
Many everyday objects around us contain free-form parts whose natural modeling involves
multi-sided (i.e., non-four-sided) surfaces. The mathematical representation of such
patches is still an active area in CAGD, and although there are many approaches,
no genuinely $n$-sided formulation has been standardized yet.

Commercial CAD/CAM systems and manufacturing applications, on the other hand, only accept data
in common formats, such as tensor product NURBS surfaces. For this reason, it is currently
standard practice to convert multi-sided surfaces into  a CAD-compatible representation
by either (i)~approximating it with a larger quadrilateral, \emph{trimming} away the excess
at the boundaries, or (ii)~\emph{splitting} it into smaller four-sided patches.
{\bf trim/split kep?}

Neither of these techniques is perfect. Both are just approximations of the original multi-sided
patch, and with trimming even $C^0$ continuity to adjacent patches cannot be ensured, while in the
splitting scheme the dividing curves in the interior weaken the overall continuity of the surface.

Ideally, we would like to have an $n$-sided patch that
\begin{enumerate}[i)]
\item can be used for design (has intuitive controls),
\item can be attached to adjacent patches with $G^1$ or higher continuity,
\item and can also be represented \emph{exactly} as a tensor product NURBS surface.
\end{enumerate}

There \emph{is} such an alternative:
if we use a multi-sided surface representation that is a rational polynomial,
then it can be converted into NURBS form, without either changing the surface or harming continuity.
The result will be a trimmed surface, but one where the trimming curves are exact.
There have been a few publications about this approach over the years;
in this paper we look at the most promising constructions, analyze and compare them,
and also give solutions to some of the issues we have encountered.

In the following, we are going to talk about
S-patches~\cite{spatch1},
Warren's patch~\cite{warren} based on B\'ezier triangles,
Kato's transfinite patch~\cite{kato},
and also propose a variation of the Charrot--Gregory patch~\cite{charrot}.
Finally, a general discussion about the aforementioned methods closes the paper.

\vspace{10pt}
\noindent\underline{S-patch:}\vspace{0.2em}\newline
The S-patch of Loop \& DeRose~\cite{spatch1} is a generalization of the B\'ezier triangle, or,
more precisely, a B\'ezier simplex mapping from $(n-1)$D to 3D, where the $n$ coordinates of the
domain are supplied by generalized barycentric coordinates. The \emph{depth} ($d$) of an S-patch
is the number of deCasteljau steps it takes to evaluate a surface point, i.e., something similar
to the degree of a B\'ezier triangle (but not the degree of the S-patch itself).
S-patches have many nice properties, and are known to be convertable into tensor product
rational B\'ezier surfaces of degree $d(n-2)$.

One drawback of this representation is its large number of control points, which renders it
inconvenient for interactive design. For example, a five-sided patch of depth 5 has 126 control
points, while a 4-sided tensor product patch has only 36. A possible workaround is to use a
$G^1$ \emph{frame} for design that defines the tangent planes at the boundaries.
After increasing the depth by 3, these boundary constraints can be interpolated~\cite{spatch2},
and the remaining interior control points can be set by some heuristic to generate a smooth
surface~\cite{salvi-kepaf}, see Figure~\ref{fig:spatch}.
\begin{figure}
  \centering
  \includegraphics[width = 0.3\textwidth]{images/5-5-bezier-ribbon.png}
  \hfill
  \includegraphics[width = 0.3\textwidth]{images/5-5-cnet-ribbon.png}
  \hfill
  \includegraphics[width = 0.3\textwidth]{images/5-5-cnet-full.png}
  \caption{Creating an S-patch based on a $G^1$ frame.}
  \label{fig:spatch}
\end{figure}

The CAD-compatible conversion presented in~\cite{spatch1} is a two-step process:
first convert the surface to a
four-sided S-patch, and then to a tensor product patch. The first step is based on the
composition of B\'ezier simplexes~\cite{simplex1}, which has very high complexity.
Even using a more efficient algorithm~\cite{simplex2}, converting a modest-sized S-patch
still requires minutes of computation on today's machines~\cite{salvi-wait}.

Here we propose an alternative conversion process. Since a B\'ezier simplex is just a polynomial,
the only problem is how to express the generalized barycentric coordinates as a rational polynomial
of the $(u,v)$ parameters on the 2D domain. Using Wachspress coordinates over a regular $n$-sided
polygon, the barycentric coordinates $\{\lambda_i\}$ are expressed as
\begin{equation}
  \label{eq:wachspress}
  \lambda_i(u,v) = \prod_{\substack{j=1\\j\notin\{i-1,i\}}}^nh_j(u,v) \quad\bigg/\quad
                   \sum_{k=1}^n\prod_{\substack{j=1\\j\notin\{k-1,k\}}}^nh_j(u,v),
\end{equation}
where the indexing is cyclic, and $h_j(u,v)$ is some distance
from the $j$-th side of the domain polygon. The implicit
equation of the line containing this side is such a distance function, and is a linear polynomial,
so the Wachspress coordinates can be expressed as rational polynomials of degree $n-2$.

Using this method, the B\'ezier control points of the tensor product representation
can be located by straightforward computation, which takes only milliseconds.

\vspace{10pt}
\noindent\underline{Warren's patch:}\vspace{0.2em}\newline
Blabla
\begin{figure}[h!]
  \begin{subfigure}{0.45\textwidth}
    \centering
    \includegraphics[width = 0.6\textwidth]{images/warren-cnet.png}
    \caption{Creating a 5-sided patch}
    \label{fig:warren-cnet}
  \end{subfigure}
  \begin{subfigure}{0.45\textwidth}
    \centering
    \includegraphics[width = 0.6\textwidth]{images/warren-quad.png}
    \caption{Degenerate tensor product patch}
    \label{fig:warren-quad}
  \end{subfigure}
  \caption{Warren's patch.}
  \label{fig:warren}
\end{figure}

\vspace{10pt}
\noindent\underline{Kato's patch:}\vspace{0.2em}\newline
Blabla

\vspace{10pt}
\noindent\underline{Charrot--Gregory patch:}\vspace{0.2em}\newline
Blabla
\begin{figure}[h!]
  \begin{subfigure}{0.23\textwidth}
    \centering
    \includegraphics[width = 0.93\textwidth]{images/trebol3-cnet.png}
    \caption{Frame and control net}
    \label{fig:trebol-cnet}
  \end{subfigure}
  \begin{subfigure}{0.23\textwidth}
    \includegraphics[width = \textwidth]{images/trebol3-contour.jpg}
    \caption{Contouring}
    \label{fig:trebol-contour}
  \end{subfigure}
  \begin{subfigure}{0.23\textwidth}
    \includegraphics[width = \textwidth]{images/trebol3-mean-iso.jpg}
    \caption{Mean curvature}
    \label{fig:trebol-mean}
  \end{subfigure}
  \begin{subfigure}{0.23\textwidth}
    \includegraphics[width = \textwidth]{images/trebol3-zebra.jpg}
    \caption{Isophote lines}
    \label{fig:trebol-iso}
  \end{subfigure}
  \caption{Conversion of Charrot--Gregory patches. \bf kar hogy haromoldaluak vannak benne}
  \label{fig:trebol}
\end{figure}

\vspace{10pt}
\noindent\underline{Discussion:}\vspace{0.2em}\newline
Blabla
\begin{figure}[h!]
  \centering
  \includegraphics[width = 0.9\textwidth]{images/rotations.png}
  \caption{Effects of rotating the multi-sided domain.}
  \label{fig:rotations}
\end{figure}
\begin{figure}[h!]
  \centering
  \begin{subfigure}{0.3\textwidth}
    \includegraphics[width = \textwidth]{images/8sided-cnet.png}
    \caption{Frame and control net}
    \label{fig:8sided-cnet}
  \end{subfigure}
  \hspace{3cm}
  \begin{subfigure}{0.3\textwidth}
    \includegraphics[width = \textwidth]{images/8sided-iso.png}
    \caption{Isophote lines}
    \label{fig:8sided-iso}
  \end{subfigure}

  \caption{Using a larger multi-sided domain.}
  \label{fig:8sided}
\end{figure}

\vspace{1em}
\noindent\underline{Conclusions:}\vspace{0.2em}\newline
Blabla

\vspace{1em}
\noindent\underline{Acknowledgement:}\vspace{0.2em}\newline
This project has been supported by the Hungarian Scientific Research Fund (OTKA, No.~124727)
and the National Research, Development and Innovation Fund
(TUDFO/51757/2019-ITM, Thematic Excellence Program).

\vspace{1em}
\noindent\underline{References:}\vspace{-1.9em}\newline
\renewcommand{\section}[2]{}
\begin{thebibliography}{99}

\bibitem{charrot} Charrot, P.; Gregory, J.~A.: A pentagonal surface patch for computer aided geometric design, Computer Aided Geometric Design, 1(1), 1984, 87--94.\\\ULurl{https://doi.org/10.1016/0167-8396(84)90006-2}
\bibitem{simplex1} DeRose, T.~D.: Composing B\'ezier simplexes, ACM Transactions on Graphics, 7(3), 1988, 198--221.\\\ULurl{https://doi.org/10.1145/44479.44482}
\bibitem{simplex2} DeRose, T.~D.; Goldman, R.~N.; Hagen, H.; Mann, S.: Functional composition algorithms via blossoming, ACM Transactions on Graphics, 12(2), 1993, 113--135.\\\ULurl{https://doi.org/10.1145/151280.151290}
\bibitem{floater} Floater, M.~S.: Generalized barycentric coordinates and applications, Acta Numerica, 24, 2015, 161--214.
\bibitem{kato} Kato, K.: Generation of $n$-sided surface patches with holes, Computer-Aided Design, 23(10), 1991, 676--683. \ULurl{https://doi.org/10.1016/0010-4485(91)90020-W}
\bibitem{toric} Krasauskas, R.: Toric surface patches, Advances in Computational Mathematics, 17(1), 2002, 89--113. \ULurl{https://doi.org/10.1023/A:1015289823859}
\bibitem{spatch1} Loop, Ch.~T.; DeRose, T.~D.: A multisided generalization of B\'ezier surfaces, ACM Transactions on Graphics, 8(3), 1989, 204--234. \ULurl{https://doi.org/10.1145/77055.77059}
\bibitem{spatch2} Loop, Ch.~T.; DeRose, T.~D.: Generalized B-spline surfaces of arbitrary topology, ACM, 17th Conference on Computer Graphics and Interactive Techniques, 1990, 347--356.\\\ULurl{https://doi.org/10.1145/97879.97917}
\bibitem{salvi-kepaf} Salvi, P.: $G^1$ hole filling with S-patches made easy, Proceedings of the 12th Conference of the Hungarian Association for Image Processing and Pattern Recognition, 2019, 1--8.
\bibitem{salvi-wait} Salvi, P.: On the CAD-compatible conversion of S-patches, Proceedings of the Workshop on the Advances of Information Technology, 2019, 72--76.
\bibitem{transfinite} V\'arady, T.; Rockwood, A.; Salvi, P.: Transfinite surface interpolation over irregular $n$-sided domains, Computer-Aided Design, 43(11), 2011, 1330--1340. \ULurl{https://doi.org/10.1016/j.cad.2011.08.028}
\bibitem{warren} Warren, J.: Creating multisided rational B\'ezier surfaces using base points, ACM Transactions on Graphics, 11(2), 1992, 127--139. \ULurl{https://doi.org/10.1145/130826.130828}

\end{thebibliography}

\end{document}
