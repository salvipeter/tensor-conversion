\documentclass[9pt,academicons]{article}

\usepackage{CADA}
\usepackage{color}
\usepackage[table]{xcolor}
\definecolor{light-gray}{gray}{0.8}
\definecolor{light-light-gray}{gray}{0.9}

\title{Notes on the CAD-Compatible Conversion of Multi-Sided Surfaces}

\begin{document}


\maketitle

\authorSection{
	\anAuthor{P\'eter Salvi}{0000-0003-2456-2051}{1},
	\anAuthor{Tam\'as V\'arady}{0000-0001-9547-6498}{2},
	\anAuthor{Alyn Rockwood}{}{3}
}

\affiliationSection{
	\anAffiliation{1}{Budapest University of Technology and Economics}{salvi@iit.bme.hu}
	\anAffiliation{2}{Budapest University of Technology and Economics}{varady@iit.bme.hu}
	\anAffiliation{3}{Boulder Graphics LLC, USA}{alynrock@gmail.com}
}

\correspondingAuthor{P\'eter Salvi}{salvi@iit.bme.hu}

\abstract{
We investigate genuine multi-sided surface representations that can be
converted into standard tensor product format, such as NURBS.  Multi-sided
patches offer versatility in shape design, and permit smooth, watertight connections
between adjacent patches using prescribed cross-derivatives; however, they can hardly be
utilized by the majority of CAD/CAM systems, which are capable to handle only
standard data formats.
On the other hand, tensor product surfaces are sometimes too rigid for shape design,
and trimmed patches can only be smoothly connected using user-defined tolerances.
% Tensor product surfaces are more rigid for shape design,
% and smoothly connecting trimmed surface patches is tolerance driven and
% generally requires extra effort.
This motivates searching for schemes that benefit from both representations.

We analyze four multi-sided surface representations that allow precise
conversion into tensor product format, namely S-patches, Warren's patch,
Kato's patch and a variant of the Charrot--Gregory patch. We compare these schemes
from various aspects; in particular, we deal with surface equations, singularities,
degrees of the converted surfaces, the control structure built and the computational
efficiency of the conversion. Several examples help to gain deeper insights into
the problem.
}

\keywords{multi-sided surfaces, trimmed surfaces, S-patch, Charrot--Gregory patch} 

\doi{10.14733/cadaps.2021.aaa-bbb}

\section{INTRODUCTION}
\label{sec:intro}
The majority of aesthetic objects is represented by free-form shapes, and modeling these
naturally involves multi-sided (i.e., non-four-sided) surfaces, as well.  The
mathematical representation of such patches is still an active area in CAGD, and although
a great variety of approaches have been published, none of the genuine $n$-sided
formulations have been standardized so far.

On the other hand, commercial CAD/CAM systems and related application programs only accept
data in standard formats, such as tensor product NURBS surfaces. For this reason, it is a
widely applied practice to convert multi-sided surfaces into a CAD-compatible
representation either by (i)~approximating them with larger quadrilaterals,
\emph{trimming} away the exterior part beyond the boundaries, or (ii)~\emph{splitting}
them into smaller four-sided patches.

Both techniques have their deficiencies. They only approximate the original multi-sided
surface, and trimming -- in general -- cannot ensure even $C^0$ continuity between the
adjacent patches. In the splitting scheme the subdividing curves in the interior
weaken the overall continuity of the surface.

Ideally, we would like to have an $n$-sided patch that:
\begin{enumerate}[label=\roman*),leftmargin=3\parindent]
\item can be used for design (has intuitive controls),
\item can be attached to adjacent patches with $G^1$ or higher continuity,
\item and can also be represented \emph{accurately} as a tensor product NURBS surface.
\end{enumerate}

The above problem can be resolved, if the multi-sided surfaces can be represented as rational
polynomials of two parametric variables. Then they can be directly converted into
NURBS form, without either changing the surfaces or harming continuity. The result will be
a collection of \emph{watertight trimmed surfaces}.
 
Some of the well-known multi-sided schemes allow computing a trimmed bi-parametric
representation. Our goal in this paper is to review these and discuss the difficulties of
the conversion process. We are going to provide further insights into specific
computational and geometric problems that have not been discussed elsewhere and are
useful for analyzing the ``pros and cons'' of these representations.

The paper is structured as follows. In Section~\ref{sec:multisided},
we review four natively multi-sided schemes, and investigate their conversions to NURBS.
In Section~\ref{sec:discussion}, issues of the conversion process are discussed in details,
with emphasis on the quality of the generated control network, and its relation to singularities.
Some more complex test examples and comparisons follow in Section~\ref{sec:tests},
showing actual high-degree examples with control grids,
and a summary of our findings concludes the paper.

\section{MULTI-SIDED SURFACES}
\label{sec:multisided}
In the following, we are going to review four multi-sided representations:
the S-patch~\cite{Loop:1989}, and the surface schemes of Warren~\cite{Warren:1992},
Kato~\cite{Kato:1991}, and Charrot \& Gregory~\cite{Charrot:1984}.
Our conversion approach is presented in
Sections~\ref{subsubsec:parameters}--\ref{subsubsec:conversion}
specifically for S-patches, but it will be adapted in later sections to other representations.

\subsection{S-Patch}
\label{subsec:spatch}
The S-patch of Loop \& DeRose~\cite{Loop:1989} is a generalization of the B\'ezier triangle, or,
more precisely, a B\'ezier simplex mapping from $(n-1)$D to 3D, where the $n$ coordinates of the
domain are supplied by generalized barycentric coordinates. The \emph{depth} ($d$) of an S-patch
is the number of deCasteljau steps it takes to evaluate a surface point, i.e., something similar
to the degree of a B\'ezier triangle (but not the degree of the S-patch itself).
S-patches have many nice properties, and are known to be convertable into tensor product
rational B\'ezier surfaces of degree $d(n-2)$.

An $n$-sided S-patch is defined over a regular $n$-gon, parameterized by generalized
barycentric coordinates $\mathbf{\lambda}=(\lambda_1,\dots,\lambda_N)$~\cite{Hormann:2017}.
Its control
points $\{P_\mathbf{s}\}$ are labeled by $n$ non-negative integers $\mathbf{s}=(s_1,\dots,s_n)$,
whose sum is the depth of the surface. Then points on the surface are given by
the equation
\begin{equation}
  \label{eq:spatch}
  S(\mathbf{\lambda})=\sum_{\mathbf{s}}P_\mathbf{s}\cdot B_\mathbf{s}^d(\mathbf{\lambda})
  =\sum_{\mathbf{s}}P_\mathbf{s}\cdot {n\choose\mathbf{s}}\cdot\prod_{i=1}^n\lambda_i^{s_i},
\end{equation}
where $B_\mathbf{s}^d(\mathbf{\lambda})$ are Bernstein polynomials with multinomial coefficients,
and the sum is computed over all possible values of $\mathbf{s}$.

One drawback of this representation is its large number of control points, which renders it
inconvenient for interactive design. For example, a five-sided patch of depth 5 has 126 control
points, while a 4-sided tensor product patch has only 36. A possible workaround is to use a
$G^1$ \emph{frame} for design that defines the tangent planes at the boundaries.
After increasing the depth by 3, these boundary constraints can be interpolated~\cite{Loop:1990},
and the remaining interior control points can be set by some heuristic to generate a smooth
surface~\cite{Salvi:2019:KEPAF}, see Figure~\ref{fig:spatch}.
\begin{figure}[h!]
  {
    \begin{subfigure}{0.3\textwidth}
      \centering
      \includegraphics[width = 0.95\textwidth]{images/5-5-bezier-ribbon.png}
      \caption{$G^1$ frame}
      \label{fig:spatch-ribbon}
    \end{subfigure}
    \hfill
    \begin{subfigure}{0.3\textwidth}
      \centering
      \includegraphics[width = 0.95\textwidth]{images/5-5-cnet-ribbon.png}
      \caption{Control points by $G^1$ constraints}
      \label{fig:spatch-cnet-ribbon}
    \end{subfigure}
    \hfill
    \begin{subfigure}{0.3\textwidth}
      \centering
      \includegraphics[width = 0.95\textwidth]{images/5-5-cnet-full.png}
      \caption{All control points}
      \label{fig:spatch-cnet-full}
    \end{subfigure}
  }
  \caption{Creating an S-patch from a $G^1$ frame.}
  \label{fig:spatch}
\end{figure}

The CAD-compatible conversion presented in~\cite{Loop:1989} is a two-step process:
first convert the surface into a
four-sided S-patch, and then to a tensor product patch. The first step is based on the
composition of B\'ezier simplexes~\cite{DeRose:1988}, which has very high complexity.
Even using a more efficient algorithm~\cite{DeRose:1993}, converting a modest-sized S-patch
still requires minutes of computation on today's machines~\cite{Salvi:2019:WAIT}.

\subsubsection{Parameterization Using Implicit Line Equations}
\label{subsubsec:parameters}
Here we propose an alternative conversion process. Since a B\'ezier simplex is just a polynomial,
the only problem is how to express the generalized barycentric coordinates as a rational polynomial
of the $(u,v)$ parameters on the 2D domain. Using Wachspress coordinates,
$\{\lambda_i\}$ can be expressed as
\begin{equation}
  \label{eq:wachspress}
  \lambda_i(u,v) = \prod_{\substack{j=1\\j\notin\{i-1,i\}}}^nh_j(u,v) \quad\bigg/\quad
                   \sum_{k=1}^n\prod_{\substack{j=1\\j\notin\{k-1,k\}}}^nh_j(u,v),
\end{equation}
where the indexing is cyclic, and $h_j(u,v)$ is a distance function
from the $j$-th side of the domain polygon. For the sake of brevity,
we will use the notation
\begin{equation}
  H_J^k(u,v)=\prod_{\substack{j=1\\j\notin{J}}}^nh_j^k(u,v),
\end{equation}
so $\lambda_i(u,v)=H_{i-1,i}^1(u,v)/\sum_{k=1}^nH_{k-1,k}^1(u,v)$.

The distance function should vanish at the base domain edge, and increase monotonically
as we get farther from it.
The implicit
equation of the line containing the base edge is suitable for this purpose,
and is also a linear polynomial,
so the Wachspress coordinates can be expressed as rational polynomials of degree $n-2$. We
normalize the distances such that they take on the value 1 at vertices adjacent to the side.

Formally, let
\begin{equation}
  V_i=\left(\frac{1}{2}+\frac{1}{2}\cos(2\pi\cdot i/n),
            \frac{1}{2}+\frac{1}{2}\sin(2\pi\cdot i/n)\right)
\end{equation}
denote the vertices of the regular $n$-sided domain inside the $[0,1]\times[0,1]$ square.
Since a line $L$ is defined by the implicit equation $L=Au+Bv+C=0$, the coefficients can
be unambiguously defined by three equations. We define $n$ lines
$L_i$ ($i=1\dots n$) by the following constraints:
\begin{align}
  L_i(V_{i-1})&=L_i(V_i)=0, & L_i(V_{i-2})&=L_i(V_{i+1})=1.
\end{align}
(Note that because of the symmetry of the regular polygon, the above four equations take
only three degrees of freedom.) The distance function is then defined as $h_i(u,v)=L_i(u,v)$.

\subsubsection{Conversion to NURBS}
\label{subsubsec:conversion}
With the above definition of generalized barycentric coordinates,
the patch equation~(\ref{eq:spatch}) becomes a rational vector polynomial in $u$ and $v$.
Since $|\mathbf{s}|=\sum_{i=1}^ns_i=d$, all terms of the sum have the same denominator,
\begin{equation}
  \left(\sum_{k=1}^nH_{k-1,k}^1(u,v)\right)^d,
\end{equation}
which is a polynomial of degree $d(n-2)$. It is easy to see that the rational degree $\delta$ of
the whole patch is also the same.

This means that we can represent an $n$-sided S-patch of depth $d$ by a rational B\'ezier
surface of $\delta\times\delta$ degrees. In order to determine the control point positions,
we still need to change from the power basis to the Bernstein basis.

Assuming that the coefficients of a bi-$\delta$-degree polynomial $p(u,v)$
are given in a matrix $M$ such that
\begin{equation}
  p(u,v)=
  \left[\begin{array}{ccccc}1&u&u^2&\dots&u^\delta\end{array}\right]
  M
  \left[\begin{array}{ccccc}1&v&v^2&\dots&v^\delta\end{array}\right]^\top,
\end{equation}
the B\'ezier coefficients are computed as $N=C^\top MC$, where $C=\{c_{ij}\}$ is
the upper triangle matrix with elements
\begin{equation}
  \begin{aligned}
    c_{ij}&={j\choose i}\bigg/{\delta\choose i}, & i,j&=0\dots\delta, & i&\leq j.
  \end{aligned}
\end{equation}
Then the same polynomial is expressed as
\begin{equation}
  p(u,v)=
  \left[\begin{array}{cccc}B_0^\delta(u)&B_1^\delta(u)&\dots&B_\delta^\delta(u)\end{array}\right]
  N
  \left[\begin{array}{cccc}B_0^\delta(v)&B_1^\delta(v)&\dots&B_\delta^\delta(v)\end{array}\right]^\top.
\end{equation}
Control point positions and the corresponding weights are computed as homogeneous coordinates,
by calculating the coefficients for the numerator and denominator of Eq.~(\ref{eq:spatch})
separately, and assigning the latter as the extra \emph{weight} coordinate.

With this method, the B\'ezier control points of the tensor product representation
can be located by straightforward computation, which takes only milliseconds.

\subsection{Warren's Patch}
\label{subsec:warren}
Warren~\cite{Warren:1992} created multi-sided patches from B\'ezier triangles by assigning $0/0$
\emph{base points} to some of the control points, essentially cutting off the corners, and thus
creating 5- and 6-sided surfaces. A simple conversion to a (degenerate) tensor product form is
also shown in the paper.

A nice property of this patch is that the ``remaining'' control points define the behavior of the
boundary in the same way as in a normal B\'ezier triangle, i.e., the first control row defines
its position as a B\'ezier curve, the second its first derivatives etc.
\begin{figure}[h!]
  \begin{subfigure}{0.50\textwidth}
    \centering
    \includegraphics[width = 0.4\textwidth]{images/warren-cnet.png}
    \caption{Control points}
    \label{fig:warren-cnet}
  \end{subfigure}
  \begin{subfigure}{0.50\textwidth}
    \centering
    \includegraphics[width = 0.4\textwidth]{images/warren-quad.png}
    \caption{Conversion to NURBS}
    \label{fig:warren-cnet}
  \end{subfigure}
  \caption{Warren's 5-sided patch.}
  \label{fig:warren}
\end{figure}

Note, however, that not all degree configurations are available. A 6-sided patch with degree-$d$
boundaries can be created from a triangle of degree $3d$, but due to its asymmetric construction,
a 5-sided patch cannot have boundaries of the same degree. Moreover, using control points with zero
weight is not a standard practice, and is not supported by many systems. Meshing
also presents a problem, as a uniform grid on the domain would result in distorted triangles
(the ``trimmed'' sides correspond to corners).

\subsection{Kato's Patch}
\label{subsec:kato}
Kato~\cite{Kato:1991} proposed a surface defined as the transfinite interpolation of boundary
curves with cross-derivatives. When these boundary constraints are given as a $G^1$ frame (and hence
are polynomial), the whole patch may become polynomial. The tricky part is the parameterization:
this representation uses two local parameters, a \emph{side parameter} $s_i$
that takes on values from 0 to 1 as it sweeps from one adjacent side to the other,
and a \emph{distance parameter} $h_i$ that vanishes on the base side $i$, see
Figure~\ref{fig:parameters}.
\begin{figure}[b!]
  \begin{subfigure}{0.30\textwidth}
    \includegraphics[width = \textwidth]{images/domain.pdf}
    \caption{Domain}
    \label{fig:domain}
  \end{subfigure}
  \hfill
  \begin{subfigure}{0.3\textwidth}
    \begin{minipage}[b][5cm][b]{\textwidth}
      \centering
      \includegraphics[width = \textwidth]{images/s-params.pdf}
      \vspace*{-2mm}
    \end{minipage}
    \caption{$s_i$ isolines for two sides}
    \label{fig:s}
  \end{subfigure}
  \hfill
  \begin{subfigure}{0.30\textwidth}
    \begin{minipage}[b][5cm][b]{\textwidth}
      \centering
      \includegraphics[width = \textwidth]{images/h-params.pdf}
      \vspace*{-2mm}
    \end{minipage}
    \caption{$h_i$ isolines for two sides}
    \label{fig:h}
  \end{subfigure}
  \caption{Parameterization.}
  \label{fig:parameters}
\end{figure}

The distance parameters can be computed the same way as in the creation of S-patches, and for side
parameters we can use $s_i=h_{i-1}/(h_{i-1}+h_{i+1})$, which gives a rational polynomial
representation.

Kato's patch is easily extendable to handle $G^2$ continuity, if the side constraints have 3
control rows (so the cross-degree $d^\perp$ is 2 instead of 1).
Then the whole surface becomes a rational tensor product B\'ezier patch of
degree $nd+(n-1)(d^\perp+1)+d^\perp$.

\subsection{Charrot--Gregory Patch}
\label{subsec:charrot}
The same idea can be used to convert Charrot--Gregory patches~\cite{Charrot:1984} that use
only side parameters. (Note that on a regular domain the $s_i$ parameters defined above
will be the same as the radial parameterization in the original paper.)
The input is given as a $G^1$ frame; the converted patch is of degree $nd+2(n-2)$.

For triangular surfaces, we can use the $h_{i-1}$ parameter as a side parameter for the $i$-th
side instead of $s_i$, thereby reducing the overall degree to $d+3$.
\begin{figure}[h!]
  \begin{subfigure}{.24\textwidth}
    \centering
    \includegraphics[height=.2\textheight]{images/trebol3-cnet.png}
    \caption{Control points}
    \label{fig:trebol-cnet}
  \end{subfigure}
  \hfill
  \begin{subfigure}{.24\textwidth}
    \centering
    \includegraphics[height=.207\textheight]{images/trebol3-contour.jpg}
    \caption{Contours}
    \label{fig:trebol-contours}
  \end{subfigure}
  \hfill
  \begin{subfigure}{.24\textwidth}
    \centering
    \includegraphics[height=.207\textheight]{images/trebol3-mean-iso.jpg}
    \caption{Mean curvature}
    \label{fig:trebol-mean}
  \end{subfigure}
  \hfill
  \begin{subfigure}{.24\textwidth}
    \centering
    \includegraphics[height=.207\textheight]{images/trebol3-zebra.jpg}
    \caption{Zebra map}
    \label{fig:trebol-zebra}
  \end{subfigure}
  \caption{Trebol model}
  \label{fig:trebol}
\end{figure}

\section{DISCUSSION}
\label{sec:discussion}
One aspect of the tensor product conversion we have not touched on before is the \emph{quality}
of the control net. Aside from Warren's patch, which has singular control points,
all the other representations have singularities in or outside their domains.
When a singular point is close to the domain of the tensor product patch (i.e., the unit square),
the control points in the vicinity show erratic behavior.

Kato's patch is singular at the corner vertices, and the S-patch is singular on the circle that goes
through the intersections of the lines containing the domain edges; however, the Charrot--Gregory patch
is singular only at these intersection points. Practically this means that excluding the triangular
S-patch (which is not rational) and the Charrot--Gregory patch for $n\leq6$ (where singularities are
relatively far away), all of these converted tensor product surfaces are likely to have
badly oscillating control points (possibly tending to infinity), which may lead to numerical issues.

We present a solution to this problem. Normally the multi-sided domain is inside the unit
square (Figure~\ref{fig:parameters}), so that the trimming curves will be inside the
surface, but if we lift this constraint, we can create a larger multi-sided domain,
thereby separating the unit square from the singularities. This means that the actual ``trimmed''
region will be outside the standard $[0,1]^2$ domain; this may not be supported by some
applications, but the control structure will be close to the surface.
\begin{figure}[h!]
  \begin{subfigure}{.3\textwidth}
    \centering
    \includegraphics[width=\textwidth]{images/8sided-1.png}
    \caption{Default domain}
    \label{fig:8sided-default}
  \end{subfigure}
  \hfill
  \begin{subfigure}{.3\textwidth}
    \centering
    \includegraphics[width=\textwidth]{images/8sided-2.png}
    \caption{Enlarged domain}
    \label{fig:8sided-enlarged}
  \end{subfigure}
  \hfill
  \begin{subfigure}{.3\textwidth}
    \centering
    \includegraphics[width=.883\textwidth]{images/8sided-3.png}
    \caption{Isophote lines}
    \label{fig:8sided-iso}
  \end{subfigure}
  \caption{An 8-sided Charrot--Gregory patch.}
  \label{fig:8sided}
\end{figure}

\begin{table}[h!]
  \centering
  \begin{tabular}{c|c|c|c|c}
    $n$ & S-patch~\cite{Loop:1989} & Warren~\cite{Warren:1992} & Kato~\cite{Kato:1991} & Charrot--Gregory~\cite{Charrot:1984} \\ \hline
    3 & \multicolumn{2}{c|}{$d[+3]$ (both B\'ezier triangles)} & \cellcolor{light-gray}$3d+5$ & $d+3$ \\ \hline
    5 & \cellcolor{light-light-gray}$3d[+9]$ & $\approx 3d$ & \cellcolor{light-gray}$5d+9$ & $5d+6$ \\ \hline
    6 & \cellcolor{light-light-gray}$4d[+12]$ & $3d$ & \cellcolor{light-gray}$6d+11$ & $6d+8$ \\ \hline
    7+ & \cellcolor{light-gray}$(n-2)(d[+3])$ & $\qquad$N/A$\qquad$ & \cellcolor{light-gray}$nd+2n-1$ & \cellcolor{light-gray}$nd+2n-4$ \\ \hline
  \end{tabular}
  \caption{Rational polynomial degrees of the converted surfaces for different number of sides,
    assuming boundary curves of degree $d$. Gray cells indicate that the surface is susceptible to
    the singularity issue.
    For S-patches, the number in brackets is applied
    when the surface is generated by a degree-$d$ $G^1$ frame.}
  \label{tab:degrees}
\end{table}

Table~\ref{tab:degrees} summarizes the degrees of all four representations.
% Note that triangular S-patches and Warren patches are both B\'ezier triangles.
It can be seen that these patches
have relatively high degrees, in particular when the number of sides and the degree of the
boundaries are raised. While Warren's patch outperforms the others in this respect,
the use of base points somewhat limits its usability in CAD systems.
Kato's surface always has singularities, and its degree is fairly high, but it is the only
construction where $G^2$ continuity can be easily achieved. We found that while the
Charrot--Gregory patch has a slightly higher degree than the S-patch, it has much lower
computational cost in its multi-sided form, and have no control net quality problems for 5-
and 6-sided configurations.

\section{TEST CASES}
\label{sec:tests}
Blah blah blah

\section{CONCLUSIONS}
In the full paper we are going to give representation details, and
further analyze the difficulties of the conversion process.
The computational efficiency of the proposed procedures
and the avoidance of wiggling control structures
will be the focal part of our discussion,
together with several comparative examples.

\section*{ACKNOWLEDGEMENTS}
This project has been supported by the Hungarian Scientific Research Fund (OTKA, No.~124727).

\section*{ORCID}
\orcid{P\'eter Salvi}{0000-0003-2456-2051}
\orcid{Tam\'as V\'arady}{0000-0001-9547-6498}

\referenceSection
\bibliographystyle{CADA}
\bibliography{cikkek}

\bigskip
\end{document}
